\section{Blocks}
\label{sec:blocks}

\nemquote{%
A stumbling block to the pessimist is a stepping-stone to the optimist.
}{Eleanor Roosevelt}

\codenamechapterfirstword is, at its core, a blockchain.
A blockchain is an ordered collection of blocks.
Understanding the parts of a \codenamespace block is fundamental to understanding the platform's capabilities.

The layout of a block is similar to the layout of an aggregate transaction \nemrefparens{sec:transactions:aggregate}.
A block shares the same unverifiable header as an aggregate transaction, and this data is processed in the same way.
Likewise, a block has a footer of unverifiable data followed by transactions.
Unlike an aggregate transaction, a block is followed by basic - not embedded - transactions, and each transaction within a block is signed independently of the block signer\footnote{
In an aggregate transaction, the account creating the aggregate transaction must sign the transaction's data in order for it to be valid.
In a block, the block signer does not need to sign the data of any transaction contained within it.
}.
This allows any transaction satisfying all conditions to be included in any block.

None of the unverifiable footer fields need to be verified.
\texttt{Reserved} bytes are used for padding and have no intrinsic meaning.

\begin{figure}
	\nemcenterwithcaption{
		\begin{subfigure}{.33\textwidth}
			\nemmemorylayout{
				\begin{leftwordgroup}{\texttt{0x00}}
					\bitbox{4}{Size} \bitbox{4}{Reserved}
				\end{leftwordgroup} \\
				\nemmemorymultiwordbox{0x08}{Signature}{7} \\
				\nemmemorymultiwordbox{0x48}{SignerPublicKey}{3} \\
				\begin{leftwordgroup}{\texttt{0x68}}
					\bitbox{4}{Reserved} \bitbox[]{4}{}
				\end{leftwordgroup}
			}{Header (unsigned)}
		\end{subfigure}%
		\begin{subfigure}{.33\textwidth}
			\nemmemorylayout{
				\begin{leftwordgroup}{\texttt{0x6C}}
					\bitbox[br]{4}{} \bitbox{4}{V, N, T}
				\end{leftwordgroup} \\
				\nemmemorysinglewordbox{0x70}{Height} \\
				\nemmemorysinglewordbox{0x78}{Timestamp} \\
				\nemmemorysinglewordbox{0x80}{Difficulty} \\
				\nemmemorymultiwordbox{0x88}{PreviousBlockHash}{3} \\
				\nemmemorymultiwordbox{0xA8}{TransactionsHash}{3} \\
				\nemmemorymultiwordbox{0xC8}{ReceiptsHash}{3} \\
				\nemmemorymultiwordbox{0xE8}{StateHash}{3} \\
				\nemmemorymultiwordbox{0x108}{\small BeneficiaryPublicKey}{3} \\
				\begin{leftwordgroup}{\texttt{0x128}}
					\bitbox{4}{\tiny FeeMultiplier} \bitbox[]{4}{}
				\end{leftwordgroup}
			}{Verifiable data}
		\end{subfigure}%
		\begin{subfigure}{.33\textwidth}
			\nemmemorylayout{
				\begin{leftwordgroup}{\texttt{0x12C}}
					\bitbox[br]{4}{} \bitbox{4}{Reserved}
				\end{leftwordgroup} \\
				\nemmemorymultiwordboxskipped{0x130}{Transactions}{}
			}{Unverifiable footer}
		\end{subfigure}
	}{Block header binary layout}
\end{figure}

\subsection{Block Fields}

\field{Height} is the block sequence number.
The first block, called the \nind{nemesis block}, has a height of one.
Each successive block increments the height of the previous block by one.

\field{Timestamp} is the number of milliseconds that have elapsed since the nemesis block.
Each successive block must have a timestamp greater than that of the previous blocks because block time is strictly increasing.
Each network tries to keep the average time between blocks close to \textit{target block time}.

\field{Difficulty} is described in detail in \nemref{sec:blockchain:difficulty}.

\field{PreviousBlockHash} is the hash of the previous block.
This is used to guarantee that all blocks within a blockchain are linked and deterministically ordered.

\field{TransactionsHash} is the merkle root hash of the hashes of the transactions stored within the block
\footnote{This field has the same purpose as the identically named field in an aggregate transaction.}.
In order to compute this field, a merkle tree is constructed by adding each transaction hash in natural order.
The resulting root hash is assigned to this field.

\field{ReceiptsHash} is the merkle root hash of the hashes of the receipts produced while processing the block.
When a network is configured without \nemsetting{network}{enableVerifiableReceipts}, this field must be zeroed in all blocks \nemrefparens{sec:blocks:receipts}.

\field{StateHash} is the hash of the global blockchain state after processing the block.
When a network is configured without \nemsetting{network}{enableVerifiableState}, this field must be zeroed in all blocks.
Otherwise, it's calculated as described in \nemref{sec:blocks:statehash}.

\field{BeneficiaryPublicKey} is the account that will be allocated the block's beneficiary share when the network has a nonzero \nemsetting{network}{harvestBeneficiaryPercentage}.
This field can be set to any account, even one that isn't yet known to the network.
This field is set by the node owner when a new block is harvested.
If this account is set to one owned by the node owner, the node owner will share in fees paid in all blocks harvested on its node.
This, in turn, incentivizes the node owner to run a strong node with many delegated harvesters.

\field{FeeMultiplier} is a multiplier that is used to calculate the effective fee of each transaction contained within a block.
The \nemsetting{node}{minFeeMultiplier} is set by the node owner and can be used to accomplish varying goals, including maximization of profits or confirmed transactions.
Assuming a block $B$ containing a transaction $T$, the effective transaction fee can be calculated as:
$$\mathfunc{effectiveFee}(T) = \structField{B}{FeeMultiplier} \cdot \mathfunc{sizeof}(T)$$
If the effective fee is greater than the transaction \field{MaxFee}, the transaction signer keeps the difference.
Only the effective fee is deducted from the transaction signer and credited to the harvester.
Further information about fee multipliers can be found in\nemref{sec:blockchain:generation}.

\subsection{Receipts}
\label{sec:blocks:receipts}

During the execution of a block, zero or more receipts are generated.
Receipts are primarily used to communicate state changes triggered by side effects to clients.
In this way, they allow simpler clients to still be aware of complex state changes.

For example, a namespace expiration is triggered by the number of blocks that have been confirmed since the namespace was created.
While the triggering event is in the blockchain, there is no indication of this state change in the block at which the expiration occurs.
Without receipts, a client would need to keep track of all namespaces and expiration heights.
With receipts, a client merely needs to monitor for expiration receipts.

Another example is around harvest rewards.
Receipts are produced that indicate the main - not delegated - account that gets credited and any beneficiary splits.
They also communicate the amount of currency created by inflation.

Receipts are grouped into three different types of statements and collated by receipt sources.
The three types of statements are transaction, address resolution and mosaic resolution.

\subsubsection{Receipt Source}

Each part of a block that is processed is assigned a two-part block-scoped identifier.
The source (0, 0) is always used to identify block-triggered events irrespective of the number of transactions in a block.

\begin{table}[ht]
	\nemcenterwithcaption{
		\begin{tabular}{|p{0.20\linewidth}|p{0.40\linewidth}|p{0.40\linewidth}|}
			\hline
			Source & Primary Id & Secondary Id \\
			\hline
			Block & 0 & 0 \\
			Transaction & 1-based index within the block & 0 \\
			Embedded Transaction & 1-based index of containing aggregate within the block & 1-based index within the aggregate \\
			\hline
		\end{tabular}
	}{Receipt source values}
\end{table}

\subsubsection{Transaction Statement}

Transaction statements are used to group receipts that have a shared receipt source.
Each statement is composed of a receipt source and one or more receipts.
Accordingly, each receipt source that generates a receipt will have exactly one corresponding transaction statement.

\begin{figure}[H]
	\nemcenterwithcaption{
		\begin{subfigure}{.5\textwidth}
			\nemmemorylayout{
				\begin{leftwordgroup}{\texttt{0x00}}
					\bitbox{4}{Size} \bitbox{2}{V} \bitbox{2}{T}
				\end{leftwordgroup} \\
				\nemmemorysinglewordbox{0x08}{ReceiptSource} \\
				\nemmemorymultiwordboxskipped{0x10}{Receipts}{}
			}{Transaction statement binary layout}
		\end{subfigure}%
		\begin{subfigure}{.5\textwidth}
			\nemmemorylayout{
				\begin{leftwordgroup}{\texttt{0x00}}
					\bitbox{4}{Size} \bitbox{2}{V} \bitbox{2}{T}
				\end{leftwordgroup} \\
				\nemmemorymultiwordboxskipped{0x08}{Payload}{}
			}{Receipt binary layout}
		\end{subfigure}
	}{Transaction statement layout}
\end{figure}

Transaction statement data is not padded because it is only written during processing and never read, so padding yields no server performance benefit.
A transaction statement hash is constructed by concatenating and hashing all statement data excepting \field{Size} fields, which are derivable from other data.

\subsubsection{Resolution Statements}

Resolution statements are used exclusively for indicating alias resolutions.
They allow a client to always resolve an unresolved value even when it changes within a block.
Theoretically, two unresolved aliases within the same block could resolve to different values if there is an alias change between their usages.
Each statement is composed of an unresolved value and one or more resolutions in order to support this.

There are two types of resolution statements - address and mosaic - corresponding to the two types of aliases.
Although \autoref{fig:blocks:mosaicResolutionStatementLayout} illustrates the layout of a mosaic resolution statement, the layout of an address resolution statement is nearly identical.
The only difference is that the resolved and unresolved values are 25-byte addresses instead of 8-byte mosaic ids.

\begin{figure}[H]
	\nemcenterwithcaption{
		\begin{subfigure}{.5\textwidth}
			\nemmemorylayout{
				\begin{leftwordgroup}{\texttt{0x00}}
					\bitbox{4}{Size} \bitbox{2}{V} \bitbox{2}{T}
				\end{leftwordgroup} \\
				\nemmemorysinglewordbox{0x08}{\small MosaicId Unresolved} \\
				\nemmemorymultiwordboxskipped{0x10}{Resolution}{Entries}
			}{Mosaic resolution statement binary layout}
		\end{subfigure}%
		\begin{subfigure}{.5\textwidth}
			\nemmemorylayout{
				\begin{leftwordgroup}{\texttt{0x00}}
					\bitbox{8}{ReceiptSource}
				\end{leftwordgroup} \\
				\begin{leftwordgroup}{\texttt{0x08}}
					\bitbox{8}{\small MosaicId Resolved}
				\end{leftwordgroup}
			}{Resolution entry binary layout}
		\end{subfigure}
	}{Mosaic resolution statement layout\label{fig:blocks:mosaicResolutionStatementLayout}}
\end{figure}

Resolution statement data is not padded because it is only written during processing and never read, so padding yields no server performance benefit.
A resolution statement hash is constructed by concatenating and hashing all statement data excepting \field{Size} fields, which are derivable from other data.

It is important to note that a resolution statement is only produced when a resolution occurs.
If an alias is registered or changed in a block, but not used in that block, no resolution statement will be produced.
However, a resolution statement will be produced by each block that contains that alias and requires it to be resolved.

\subsubsection{Receipts Hash}
\label{sec:blocks:receiptshash}

In order to calculate a block's receipt hash, first all statements generated during block processing are collected.
Then a merkle tree is created by adding all statement hashes in the following order:

\begin{enumerate}
\item{Hashes of transaction statements ordered by receipt source.}
\item{Hashes of address resolution statements ordered by unresolved address.}
\item{Hashes of mosaic resolution statements ordered by unresolved mosaic id.}
\end{enumerate}

When a network is configured with \nemsetting{network}{enableVerifiableReceipts}, the root hash of this merkle tree is set as the block's \field{ReceiptHash}.
A client can perform a merkle proof to prove a particular statement was produced during the processing of a specific block.

\subsection{State Hash}
\label{sec:blocks:statehash}

\codenamespace stores the global blockchain state across multiple typed state repositories.
For example, the account state is stored in one repository and the multsignature state is stored in another.
Each repository is a simple key value store.
The specific repositories present in a network are determined by the transaction plugins enabled by that network.

When a network is configured with \nemsetting{network}{enableVerifiableState}, a patricia tree is created for each repository.
This produces a single hash that deterministically fingerprints each repository.
Accordingly, assuming $N$ repositories, $N$ hashes deterministically fingerprint the global blockchain state.

It is possible to store all $N$ hashes directly in each block header, but this is undesirable.
Each block header should be as small as possible because all clients, minimally, need to sync all headers to verify a chain is rooted to the nemesis block.
Additionally, adding or removing functionality could change the number of repositories ($N$) and the format of the block header.

Instead, all root hashes are concatenated\footnote{The concatenation order is fixed and determined by the repository id.} and hashed to calculate the \field{StateHash},
which is a single hash that deterministically fingerprints the global blockchain state.
\begin{align*}
\mathit{RepositoryHashes} = \: & 0 \\
\mathit{RepositoryHashes} = \: & \sum_{0\leq i \leq N} \mathfunc{concat}(\mathit{RepositoryHashes}, \mathit{RepositoryHash}_i) \\
\mathit{StateHash} = \: & H(\mathit{RepositoryHashes})
\end{align*}

\subsection{Extended Layout}

The block layout described earlier was correct with one simplification\footnote{
This is consistent with the extended layout of an aggregate transaction as well.}.
All transactions are padded so that they end on 8-byte boundaries.
This ensures that all transactions begin on 8-byte boundaries as well.
The padding bytes are never signed nor included in any hashes.

\begin{figure}[H]
	\nemcenter{
		\nemmemorylayout{
			\begin{leftwordgroup}{}
				\bitbox[br]{4}{} \bitbox{4}{Reserved}
			\end{leftwordgroup} \\
			\nemmemorymultiwordboxskipped{}{Transaction 1}{} \\
			\nemmemorysinglewordbox{}{\emph{optional} padding} \\
			\nemmemorymultiwordboxskipped{}{Transaction 2}{} \\
			\nemmemorysinglewordbox{}{\emph{optional} padding}
		}{Block transaction footer with padding}
	}
\end{figure}
