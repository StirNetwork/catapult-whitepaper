\section{Partial Transactions}
\label{sec:partials}

\nemquote{%
The whole is greater than the sum of its parts.
}{Aristotle}

\nemchapterfirstletter{B}{onded} aggregate transactions \nemrefparens{sec:transactions:aggregate} are also referred to as partial transactions.
The name \emph{partial} is fitting because the transactions have insufficient cosignatures and are unable to pass validation until more cosignatures are collected.

Support for handling partial transactions is provided by the \textit{partial transaction extension}.
If a network supports bonded aggregate transactions, this extension should be enabled on all Api and Dual nodes.

Partial transactions are synchronized among all nodes in a network that have this extension enabled.
A node passively receives partial transactions and cosignatures pushed by remote nodes.
It also periodically requests transactions and cosignatures from remote nodes via the \textit{pull partial transactions task}.
As an optimization, the requesting node indicates which transactions and cosignatures it already knows in order to avoid receiving redundant information.

When the hash lock plugin is enabled, in order for a partial transaction to be accepted by the network, a hash lock must be created and associated with the transaction.
The hash lock is essentially a bond paid in order to use the built-in cosignature collection service.
If the associated partial transaction is completed and confirmed in the blockchain before the hash lock expires, the bond is returned to the payer.
Otherwise, the bond is forfeited to the harvester of the block at which the lock expires.
This feature makes spamming the partial transactions cache more costly because it requires \nemsetting{node}{lockedFundsPerAggregate} to be paid, at least temporarily, for a partial transaction to enter the cache.

When a node receives a new partial transaction, it is pushed to the \textit{partial transaction dispatcher}.
Received cosignatures are collated with partial transactions already in the cache and are immediately rejected if there are no matching transactions
\footnote{This implies that a partial transaction must be present in the cache before any of its cosignatures can be accepted.}.
When transactions and cosignatures are received together, they are split and processed individually as above.

Compared to the normal transaction dispatcher \nemrefparens{sec:disruptor:consumers}, the partial transaction dispatcher is minimalistic.

\begin{figure}[H]
	\nemcenterwithcaption{
		\tikzstyle{consumer} = [rectangle, rounded corners, text centered, draw=black, minimum width=7cm, minimum height=0.5cm, text width=6.5cm]
		\vspace{0pt}
		\begin{tikzpicture}[>=latex,font=\small ,semithick,scale=1,node distance=1.1cm]
			% consumers
			\node (hash calculator consumer) [consumer, below of=audit consumer] {Hash Calculator Consumer};
			\node (hash check consumer) [consumer, below of=hash calculator consumer] {Hash Check Consumer};
			\node (new transactions consumer) [consumer, below of=hash check consumer] {New Transactions Consumer};

			% arrows
			\draw [semithick,->] (hash calculator consumer) -- (hash check consumer);
			\draw [semithick,->] (hash check consumer) -- (new transactions consumer);
		\end{tikzpicture}
	}{Partial transaction consumers}
\end{figure}

The hash calculator and hash check consumers work as described for the transaction dispatcher in \nemref{sec:disruptor:commonConsumers}.
The one difference is that the hash check consumer will additionally search the partial transactions cache for previously seen transactions.
The new transaction consumer has a similar purpose to the one described in \nemref{sec:disruptor:transactionConsumers}.
The difference is that it broadcasts and processes partial transactions instead of unconfirmed transactions.
Specifically, it broadcasts partial transactions to the network and then adds valid ones to the partial transactions cache.

\subsection{Partial Transaction Processing}
\label{sec:partials:processing}

The partial transactions cache contains all partial transactions that are waiting for additional cosignatures.
When a new partial transaction is received that passes all validation, it is added to the cache.
It will stay in the cache until a sufficient number of cosignatures are collected or it becomes invalid.
For example, the transaction will be purged if its associated hash lock expires.
Whenever a partial transaction is completed by collecting sufficient cosignatures, it will immediately be forwarded to the transaction dispatcher and be processed as an unconfirmed transaction.

\begin{figure}
	\nemcenterwithcaption{
		\input{chapters/partial_flowchart}
	}{Processing of a partial transaction}
\end{figure}

The partial transactions cache collates all new cosignatures with the transactions it already contains.
In order to be added to the cache, a cosignature must be new, verifiable and associated with an existing partial transaction.
It is possible for a previously accepted cosignature to become invalid, in which case it should be purged.
For example, a cosignature could become invalid if its signer was removed from a multisignature account participating in the partial transaction.
The cosignature collation process is intricate enough to handle such edge cases.

\begin{figure}
	\nemcenterwithcaption{
		\input{chapters/partial_flowchart_cosignatures}
	}{Processing of a cosignature}
	\label{fig:partials:flowchart:cosignature}
\end{figure}
