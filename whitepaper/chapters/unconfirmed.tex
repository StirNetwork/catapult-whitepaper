\section{Unconfirmed Transactions}
\label{sec:unconfirmedTransactions}

\nemquote{%
I don’t believe one reads to escape reality.
A person reads to confirm a reality he knows is there, but which he has not experienced.
}{Lawrence Durrell}

\nemchapterfirstletter{A}{ny} transaction that is not yet included in a block is called an \emph{unconfirmed transaction}.
These transactions may be valid or invalid.
Valid unconfirmed transactions are eligible for inclusion in a harvested block.
Once a transaction is added to a block that is accepted in the blockchain, it is \emph{confirmed}.

Unconfirmed transactions can arrive at a node when:
\begin{enumerate}
	\item{A client sends a new transaction directly to the node.}
	\item{A bonded aggregate transaction is completed with all requisite cosignatures and is promoted from the partial transactions cache.}
	\item{A Peer node broadcasts transactions to the node.}
	\item{A Peer node responds to the node's request for unconfirmed transactions.
	As an optimization, the requesting node indicates what transactions it already knows in order to avoid receiving redundant transactions.
	Additionally, it supplies the minimum fee multiplier \nemrefparens{sec:blockchain:generation} it uses when creating blocks.
	This prevents the remote node from returning unconfirmed transactions that will be immediately rejected by the requesting node.}
\end{enumerate}

When an unconfirmed transaction arrives at a node, it is added to the transaction disruptor \nemrefparens{sec:disruptor}.
All transactions that haven't been previously seen and pass stateless validation will be broadcast to peer nodes.
At this point, it is still possible for the node to reject the broadcast transactions because stateful validation is performed after broadcasting.
Due to different node settings, it's possible for some nodes to accept a specific unconfirmed transaction and other nodes to reject it.
For example, the nodes could have different \nemsetting{node}{minFeeMultiplier} settings.

\subsection{Unconfirmed Transactions Cache}
\index{unconfirmed transactions!cache}

When a transaction passes all validation, it is eligible for inclusion in a harvested block.
At this point, the node tries to add it to the \emph{unconfirmed transactions cache}.
This can fail for two reasons:
\begin{enumerate}
	\item{The maximum cache size configured by \nemsetting{node}{unconfirmedTransactionsCacheMaxSize} has been reached.}
	\item{The cache contains at least as many unconfirmed transactions as can be included in a single block and the new transaction is rejected by the Spam throttle}.
\end{enumerate}

Whenever new blocks are added to the blockchain, the blockchain state changes and the unconfirmed transactions cache is affected.
Although all transactions in the cache are valid at the time they were added, this doesn't guarantee that they'll be valid in perpetuity.
For example, a transaction could have already been included in a block harvested by another node or a conflicting transaction could have been added to the blockchain.
This means that transactions in the cache that were perfectly valid previously could be invalidated after changes to the blockchain state.
Additionally, when blocks with transactions are reverted, it's possible that some of those previously confirmed transactions are no longer included in any block in the new chain.
Those reverted transactions should be added to the cache.

As a result of these considerations, the entire unconfirmed transactions cache is completely rebuilt whenever the blockchain changes.
Each transaction is rechecked by the stateful validators and purged if it has become invalid or has already been included in a block.
Otherwise, it is added back to the cache.

\subsection{Spam Throttle}
\index{unconfirmed transactions!spam throttle}

The initiator of an unconfirmed transaction does not have to pay a fee to nodes holding the transaction in the unconfirmed transactions cache.
Since the cache uses valuable resources, a node must have some protection against being spammed with lots of unconfirmed transactions.
This is especially important if the node is generous and accepts zero fee transactions.

Simply limiting the number of unconfirmed transactions that a node accepts is suboptimal because normal actors should still be able to send a transaction even when a malicious actor is spamming the network.
Limiting the number of unconfirmed transactions per account is also not a good option because accounts are free to create.

\codenamespace implements a smart throttle that prevents an attacker from filling the cache completely with transactions while still letting honest actors successfully submit new unconfirmed transactions.
\nemsetting{node}{enableTransactionSpamThrottling} can be used to activate the throttle.
Assuming the cache is not full, it works in the following way:
\begin{enumerate}
	\item{If the cache contains fewer unconfirmed transactions than can be included in a single block, throttling is bypassed.}
	\item{If the new transaction is a bonded aggregate transaction, throttling is bypassed.}
	\item{Else the Spam throttle is applied.}
\end{enumerate}

Let $\mathvar{curSize}$ be the current number of transactions in the cache and $\mathvar{maxSize}$ the configured maximum size of the cache.
Also let $\mathname{rel. importance of A}$ be the relative importance of $A$, i.e. a number between 0 and 1.
If a new unconfirmed transaction $T$ with signer $A$ arrives, then the $\mathname{fair share}$ for account $A$ is calculated:
\begin{align*}
	\mathvar{maxBoostFee} &= \nemsetting{node}{transactionSpamThrottlingMaxBoostFee} \\
	\mathvar{maxFee} &= \min(\mathvar{maxBoostFee}, \structField{T}{MaxFee}) \\
	\mathname{eff. importance} &= (\mathname{rel. importance of A}) + 0.01 \cdot \frac{\mathvar{maxFee}}{\mathvar{maxBoostFee}} \\
	\mathname{fair share} &= 100 \cdot (\mathname{eff. importance}) \cdot (\mathvar{maxSize} - \mathvar{curSize}) \cdot \exp\left(-3 \frac{\mathvar{curSize}}{\mathvar{maxSize}}\right)
\end{align*}

If account $A$ already has as many transactions in the cache as its fair share, then the new transaction is rejected.
Otherwise, it is accepted.
The formula shows that an increase in a transaction's maximum fee increases the number of slots available in the cache.
Nonetheless, this mechanism for boosting the effective importance is limited by \nemsetting{node}{transactionSpamThrottlingMaxBoostFee}.

\begin{figure}[ht]
	\nemcenterwithcaption{
		\pgfplotsset{
			legend entry/.initial=,
			every axis plot post/.code={%
				\pgfkeysgetvalue{/pgfplots/legend entry}\tempValue
				\ifx\tempValue\empty
					\pgfkeysalso{/pgfplots/forget plot}%
				\else
					\expandafter\addlegendentry\expandafter{\tempValue}%
				\fi
			},
		}
		\begin{tikzpicture}[trim axis left]
			\begin{axis}[
					scale only axis,
					xlabel = \text{cache fill level [\%]},
					xticklabel={\pgfmathparse{int(\tick / 100)}\pgfmathprintnumber{\pgfmathresult}},
					scaled x ticks=false,
					ylabel = fair share,
					height = 5cm,
					width=(4 * \textwidth) / 5
				]
				\addplot[color=blue,domain=1500:10000,legend entry=$0.01\%$]{100 * 0.0001 * (10000 - x) * exp(-3 * x / 10000)};
				\addplot[color=red,domain=1500:10000,legend entry=$0.005\%$]{100 * 0.00005 * (10000 - x) * exp(-3 * x / 10000)};
				\addplot[color=green,domain=1500:10000,legend entry=$0.001\%$]{100 * 0.00001 * (10000 - x) * exp(-3 * x / 10000)};
			\end{axis}
		\end{tikzpicture}
	}{Fair share for various effective importances with max cache size = 10000}
	\label{fig:unconfirmedTransactions:fairShare}
\end{figure}

\autoref{fig:unconfirmedTransactions:fairShare} shows the fair share of slots relative to the fill level of the cache for various effective importances.
An attacker that tries to occupy many slots cannot gain much by using many accounts because the importance of each account will be very low.
The attacker can increase maximum transaction fees but that will be more costly and expend funds at a faster rate.
