\section*{Preface}
\addcontentsline{toc}{section}{Preface}

\nemquote{%
You miss 100\% of the shots you don't take.
}{Wayne Gretzky}

\nemchapterfirstletter{N}{EM} had its humble beginnings as a "call for participation" on a bitcointalk thread in January 2014.
The cryptospace had just experienced a boom at the tail end of 2013 - although nowhere near where it would go a few years later - and there was a lot of enthusiasm in the space.
NXT had just launched as one of the first PoS blockchains, and much of the early NEM community was inspired by and had connections to the NXT community. This includes all three of the remaining core developers.

Although there was some initial discussion on what to build, we quickly decided to create something new from scratch.
This allowed for more design flexibility as well as the use of high coding standards from the very beginning.
This also gave us the opportunity to contribute something new to the blockchain landscape.
As a result of lots of effort - mostly nights and weekends - this culminated in the release of NIS1 mainnet in March 2015.
We were pleased with what we built, but knew we took a few shortcuts, and continued improving it.
Eventually, we came to the realization that the original solution would need a rearchitecture to fix some central performance bottlenecks and allow faster innovation in the future.

We are grateful to TechBureau who provided support for us to build a completely new chain from scratch - \codename.
We are hopeful that this fixes many of the problems inherent in NIS1 and provides a solid foundation for future improvements and enhancements.
Our mandate was to build a high performance *blockchain* - not a DAG or dBFT based system.
In this, we think, we succeeded.

This has been a long journey for us, but we are excited to see what yet is to come and what novel things \textit{you} use \codenamespace to build.
We would like to once again thank the contributors and the many people who have inspired us\ldots

\begin{flushright}
BloodyRookie
gimre
Jaguar0625
\end{flushright}
